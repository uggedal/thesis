\documentclass[12pt,a4paper]{article}

\usepackage[utf8]{inputenc}
\usepackage[T1]{fontenc}
\usepackage[norsk]{babel}

\setcounter{secnumdepth}{-1} 

\author{Eivind Uggedal <eivindu@ifi.uio.no>}
\title{Prosjektbeskrivelse}

\begin{document}

\maketitle{}

\section{Foreløpig tittel}
\emph{Akser for sosial navigasjon på web}

\section{Innledning}

Sosial navigasjon i forhold til et nettsted kan dreie seg om å finne ``stier''
for å forflytte seg mellom ulike deler av innholdet basert på ``spor'' lagt
igjen av andre brukere. For nettsteder som Amazon, YouTube, Flickr, MySpace,
m.fl, er det sosiale aspektet ved navigasjonen ofte en sentral del av
brukeropplevelsen.

De fleste har en overveiende positiv opplevelse av både å bruke slike spor, og
den støtten som nettstedet kan gi navigasjonene gjennom de spor de selv har
lagt ut. Samtidig vet vi at innsamling og analyse av slike ``spor'' kan være
problematisk ut fra et personvernperspektiv.

\section{Mål}

Oppgaven dreier seg om å skaffe til veie bedre forståelse av slike ``akser'',
og hvordan de brukes av brukerene på ulike nettsteder. Videre skal det
implementeres en prototype av et sosialt nettsted som i utstrakt grad gjør
bruk av slike akser, og som også muligjør at brukerne selv tar del i
utviklingen av slike aksler og har en grad av kontroll over hvorledes de
benyttes.

\begin{itemize}
  \item Studie av, og beskrivelse og analyse av, en del nettsteder som er
    sterke på sosial navigasjon.
  \item Definisjon/beskrivelse av ulike akser/dimensjoner for sosial
    navigasjon.
  \item Analyse av hvordan slike akser kan brukes.
  \item Hvordan kan brukerne selv kontrollere hvilke ``spor'' de legger
    igjen og hvordan disse benyttes av andre?
  \item Utvikling av prototype et nettsted som muligjør brukerutvikling og
      --kontroll av mange slike akser.
\end{itemize}

\section{Foreløpig problemstilling}
Hvordan kan akser for sosial navigasjon brukes for å heve brukeres opplevelse
og deltagelse på web mens man samtidig legger til rette for
individers mulighet til selv å begrense graden av involvering?

\section{Metodebruk}
Det vil bli foretatt innholdsundersøkelser av relevante nettsteder samt
brukerundersøkelser for aktive brukere av nettstedene.

\section{Omfang}

Oppgavearbeidet skal tilsvare 60 studiepoeng og være fordelt over tre
semestre. Produktene av denne prosessen vil være en rapport som omhandler
sentrale tema med tilstrekkelig dybde og en prototype av et nettsted som
realiserer disse aspektene.

\section{Fremdriftsplan}

Følgende høynivå fremdriftsplan lister opp milepæler og måneden de skal være
ferdigstilt.

\begin{description}
  \item[2007.06] Grunnleggende undersøkelser omkring akser for sosial
    navigasjon slik at man har en god nok forståelse for å kunne utvikle
    en prototype i tillegg til første utkast til rapport.
  \item[2007.12] Utvikling av prototype hvor sentrale deler gjenspeiles i
    rapporten.
  \item[2008.06] Grundigere undersøkelser basert på prototypearbeid og
    ferdigstilling av rapport.
\end{description}

\end{document}
