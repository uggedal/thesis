\documentclass[12pt,a4paper]{article}

\usepackage[utf8]{inputenc}
\usepackage[T1]{fontenc}

\author{Eivind Uggedal <eivindu@ifi.uio.no>}
\title{Article Summaries}

\begin{document}

\maketitle{}

\section{Social Navigation}

Dieberger, A., Dourish, P., Hook, K., Resnick, P., (2000).
Social Navigation: Techniques for Building More Usable Systems.
\emph{interactions 7}(6), 36-45.
\break

Social navigation, when you're using information from other people to
help you make a decision in finding your way, is different from a tool based
approach like using maps or guides. Even though these ideas can be
used in our digital world, the adoption has been slow, leaving web surfers
with few indicators of other people's direct or indirect precense.
Webdevelopers should include facilities that make a user avare of other users
activities where appropriate without, infridging their privacy nor hindering
them in their work.

\end{document}
