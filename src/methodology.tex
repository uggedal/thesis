\chapter{Methodology}
\label{chapter:methodology}

This chaper will include information in how data was collected. So far this
includes content inventory/analysis.

\section{Content Analysis}

The term \emph{content analysis} is traditionally used to signify a
qualitative research method used in the social sciences.
\citet[p.~18]{krippendorff03} defines it as:
``a research technique for making replicable and valid
inferences from texts (or other meaningful matter) to the contexts of their
use''. Even though such an analysis of the contents, meanings, or effects of
communication messages have been utilized on the Web \citep{weare00} it will
not help us understand navigational mechanisms.

We turn to content analysis as the more pragramtic practice conducted within
the field of \emph{information architecture}%
\sidenote{
  Information architecture can be explained as
  \begin{inparaenum}[(a)]
    \item the structural design of shared information environments,
    \item the combination of organization, labeling, search, and navigation
      systems within web sites and intranets,
    \item the art and science of shaping information products and experiences
      to support usability and findability, and
    \item an emerging discipline and community of practice focused on bringing
      principles of design and architecture to the digital landscape
      \citep[p.~4]{morville06}.
  \end{inparaenum}
} hoping that it will help us get an better understanding of navigational
structures.
Content analysis is deployed as a technique by information architects for
helping them generate a sound and well structured web site architecture.
In it's essence a content analysis should identify the various
relationships (or lack of correlation) between a web site's content items.
It consists of two phases:
\begin{inparaenum}[(i)]
  \item a collection of a representative sample of data and
  \item an analyses of this collected data
\end{inparaenum}
\citep[pp.~241--243]{morville06}.

\subsection{Inventory}

A \emph{content inventory} is a technique for collecting data from web sites
in a structured manner. It's strength as a technique lies in it's ability of
truly informing it's receipents about a web site's content.

\subsubsection{Samping}

Content inventories are often tedious and time consuming to perform.
\citet[p.~267]{wodtke02} argues that every single bit of content needs to be
determined while \citet[p.~241]{morville06} believes a representative sample
is sufficient.

The web sites that are interesting to look at in our research are vast and
loaded with enourmous ammounts of user generated content. An all-inclusive
approach to content gethering would simply be impossible in such situations.
As a remedy to this we've decided to ignore certain parts of web sites in our
content inventories since the scope of our research is limited to navigational
constructs and only those wich have a social nature.

Our experience is that social navigation and more static navigation are
intermixed all over web sites. Often one have to use non-social forms of
navigation before social navigational options appear. Thus we could not simply
ignore navigational aims wich were non-social in our content inventory phase.
We did however eliminate the following parts of web sites under investigation:

\begin{description}
  \item[Administrative sections] where users can change their profiles or set
    their preferences.
  \item[Help pages] where FAQs, guides, and instructions are presented in a
    static manner.
  \item[Legal information] including terms of service, privacy policies, and
    copyright notices.
  \item[Content generation] facilities like uploading, categorizing, and
    editing photos, commenting, posting items, and so on%
    \sidenote{
      While there is no question about the usefullness of such content for
      providing social navigation posibilities we've found few examples where
      social navigation is used in the content generated phases themself.
      There is however a few exceptions and these will be discussed when
      appropriate. % collaborative tagging for instance
    }.
  % to be extended when more sites are investigated
\end{description}

In addition to eliminating certain form of web pages we synthesised abstract
page representations by introducing variables. Take for example a typical
social networking site. There are from thousands to several millions of
profile pages. In context of what navigational options these pages present to
us they are all esential similar. So we could introduce a variable called
\var{user-name}%
\sidenote{
  The variable notation with a dollar sign (\$) as prefix is inspired from
  variable usage in UNIX shell scripting \citep[p.~88]{kernighan84}.
}
and thereby describe all potential profile pages as: ``Profile
of \var{user-name}''.

We would however have to make sure that the one page we used in our inventory
to represen the abstract notion of a profile page was representative. To
examplify, say that a profile page included a stream of the 10 most recent
actions your friends had conducted. If the user of our collected profile page
had zero friends we would lack the navigational oppertunities such a stream
could give us in our inventory. Therefore we used only pages which provided
all possible forms of navigation as basis for abstraction.


\subsection{Analysis}

\subsection{Web Site Selection Criterias}

Instead of using content analysis as a means for improving on an existing
site's content architecture we'll be tailoring this technique to best help us
discover and understand social navigation patterns in infamous web sites which
are known to make good use of such navigational designs.

