\chapter{Methodology}
\label{chapter:methodology}

This chapter will include information in how data was collected. So far this
includes content inventory/analysis.

\section{Content Analysis}

The term \term{content analysis} is traditionally used to signify a
qualitative research method used in the social sciences.
\prequote[p.~18]{krippendorff03}{defines it as}{%
  a research technique for making replicable and valid
  inferences from texts (or other meaningful matter) to the contexts of their
  use}
Even though such an analysis of the contents, meanings, or effects of
communication messages also have been utilized on the Web \citep{weare00}
it does not seem very well suited for understanding navigational mechanisms.

We turn to content analysis as the more pragmatic practice conducted within
the field of \term{information architecture}%
\sidenote{
  Information architecture can be explained as
  \begin{inparaenum}[(i)]
    \item the structural design of shared information environments,
    \item the combination of organization, labeling, search, and navigation
      systems within web sites and intranets,
    \item the art and science of shaping information products and experiences
      to support usability and findability, and
    \item an emerging discipline and community of practice focused on bringing
      principles of design and architecture to the digital landscape
      \citep[p.~4]{morville06}.
  \end{inparaenum}
} hoping that it will help us get an better understanding of navigational
structures.
Content analysis is deployed as a technique by information architects for
helping them generate a sound and well structured web site architecture.
It's seen as a bottom-up process and
in it's essence a content analysis should identify the various
relationships (or lack of correlation) between a web site's content items.
It consists of two phases:
\begin{inparaenum}[(i)]
  \item a collection of a representative sample of data and
  \item an analysis of this collected data
\end{inparaenum}
\citep[pp.~241--243]{morville06}.

Information architects are concerned with
the system's content and
\postquote[p.~94]{batley07}{%
  need to move below the surface of the system
  interface to examine the system information itself}
We on the other hand are actually concerned with the system interface and
specifically it's navigational structures. This creates a striking
contradiction as we're not interested in content unless it can help or
guide users during their navigation. We therefore have to adapt
both our inventory and analysis process accordingly.

\subsection{Inventory}

A \term{content inventory} is a technique for collecting data from web sites
in a structured manner. It's strength as a technique lies in it's ability of
truly informing about a web site's content \citep{wodtke02}. The process of
actually conducting a content inventory can be equally rewarding as the
resulting documents \citep{veen02}.

\subsubsection{Sampling}
\label{section:methodology.content.analysis.sampling}

Content inventories are often tedious and time consuming to perform.
\citet[p.~267]{wodtke02} argues that every single bit of content needs to be
determined while \citet[p.~241]{morville06} believes a representative sample
is sufficient.

The web sites that are interesting to look at in our research are vast and
loaded with enormous amounts of user generated content. An all-inclusive
approach to content gathering would simply be impossible in such situations.
As a remedy to this we've decided to ignore certain parts of web sites in our
content inventories since the scope of our research is limited to navigational
constructs and only those which have a social nature.

Our experience is that social navigation and more static navigation are
intermixed all over web sites. Often one have to use non-social forms of
navigation before social navigational options appear. Thus we could not simply
ignore navigational aims which were non-social in our content inventory phase.
We did however eliminate the following parts of web sites under investigation:

\begin{description}
  \item[Administrative sections] where users can change their profiles or set
    their preferences.
  \item[Help pages] where FAQs, guides, and instructions are presented in a
    static manner.
  \item[Legal information] including terms of service, privacy policies, and
    copyright notices.
  \item[Content generation] facilities like uploading, categorizing, and
    editing photos, commenting, posting items, and so on%
    \sidenote[-10\onelineskip]{
      While there is no question about the usefulness of such content for
      providing social navigation possibilities we've found few examples where
      social navigation is used in the content generation phases itself.
      There is however a few exceptions and these will be discussed when
      appropriate. % collaborative tagging for instance
    }.
  % to be extended when more sites are investigated
  \item[Advertisements] from third party advertisers.
\end{description}

In addition to eliminating certain form of web pages we synthesized abstract
page representations by introducing variables. Take for example a typical
social networking site. There are from thousands to several millions of
profile pages. In context of what navigational options these pages present to
us they are all essential similar. So we could introduce a variable called
\var{user-name}%
\sidenote[-5\onelineskip]{
  The variable notation with a dollar sign (\$) as prefix is inspired from
  variable usage in UNIX shell scripting \citep[p.~88]{kernighan84}.
}
and thereby describe all potential profile pages as:
\val{Profile of \var{user-name}},

We would however have to make sure that the one page we used in our inventory
to represent the abstract notion of a profile page was representative. To
exemplify, say that a profile page included a stream of the 10 most recent
actions your friends had conducted. If the user of our collected profile page
had zero friends we would lack the navigational opportunities such a stream
could give us in our inventory. Therefore we used only pages which provided
all possible forms of navigation as basis for abstraction.

\subsubsection{Approach}

We started out on the first page that was given us when entering the web site
under investigation. From there we stepped trough each page of the web site
by following all navigational hyperlinks provided on individual pages.
We did no however frequent a web site in it's entirety, but bearing in mind
our sampling constraints and abstractions we frequented the site to full
coverage. We stopped browsing a particular page if it%
\oddsidenote{
  Either the exact same page or a page deemed to be the same by our
  variable driven abstraction method.
}
had been previously been inventoried. During the course of this browsing
each page was noted down in a table with the following characteristics:

\begin{description}
  \item[Identifier] of numerical and hierarchical form where page with id
    4.3.1 is the first child of a page with id 4.3, representing it's place in
    the navigation structure. The first page was given an id of \val{0},
    it's first descendant an id of \val{1} and so on.
  \item[Page title] as a description of what the page contains. Some of the
    web site's we surveyed had a slight ambiguity of title usage. In these
    cases we decided to collect the most representative sample%
    \sidenote{
      A choice between the \code{<title>} element in the \code{<head>}
      of the HTML document and the \code{<h1>} top level heading in-line
      the \code{<body>} portion of the document was made.
    }.
    If this resulted in unsatisfactory results we created a new title using
    the best of our abilities to make it as clear and descriptive as possible.
  \item[Link name] is either the textual name or a description of the
    contents (i.e. a graphical representation) of the hyperlink that was
    utilized to navigate to this very page.
  \item[Link location] as a description of the spatial position
    (for example: global navigation, content area, right sidebar) of the
    hyperlink used to navigate to this page.
  \item[Page URL] as an identifier for the page we visited.
\end{description}

The result was a table representing a web site's various pages and the
navigational relationships amongst them.

While we took note of the URL of each page we've decided not to display this
information. We're not convinced of it's usefulness in light of navigation
and therefore for brevity sake omitted them. We did however use them in our
inventory process as a way to identify previously collected pages.
We were able to clearly distinguish between different pages
by introducing variables into the URLs as well.

In a traditional content inventory other characteristics
is usually collected. For an example see \citet[p.~269]{wodtke02}.
As we described earlier we're only concerned with the navigational parts of
web pages. We opted to only record what we found to be useful for this
purpose. This lead to a situation where we were collecting more information
about site structure than the attributes of a site's content\dash{}as an
information architect usually would do.

\subsection{Analysis}

An analysis of the collected content follows after an inventory phase is
completed. Typically information architects use content analysis for
making decisions on what and how to improve an existing web site's content
architecture. With such an aim they look for patterns and relationships when
analyzing their content inventory. These patterns and relationships will then
suggest groupings and connections amongst separate content items
\cite[p.~243]{morville06}. For an example of further issues that information
architects tends to focus on see \citet{leise07} and his set of 11 heuristics%
\evensidenote{
  Shortly summarized as:
  collocation,
  differentiation,
  completeness,
  information scent,
  bounded horizons,
  accessibility,
  multiple access paths,
  appropriate structure,
  consistency,
  audience-relevance, and
  currency.
}
for content analysis.

\subsubsection{Approach}
% In this approach section i write in future tense.
% In the former approach section i wrote in past tense.
% This is due to the fact that inventory is completed and
% analysis is not (for Flickr).
Since our focus is dissimilar to that of most information architects' we've
had to tailor the analysis process to best help us discover and understand
patterns of social navigation in web sites. Analyzing content inventories for
such means is as far as we know not conducted before. We're therefore
exploring unknown waters and possibly have to adapt our method as we go.

We will start with our impressions from the content inventory%
\sidenote{
  As stated earlier the process of conducting a content inventory is not only
  beneficial just because of the resulting documentation one creates of a web
  site. People conducting content inventories tends to get deeply informed
  about a web sites content and structure after having exhaustively recorded
  large parts of it. 
}
and base our discussion on the findings we regard most conspicuous in
relation to social navigation. During this discussion we'll try to reference
the pages recorded in our content inventory by their identifiers.
