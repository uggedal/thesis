\chapter{Discussion}
\label{chapter:discussion}

This chapter will start with a synthesis based on the analysis of our
collected data on social navigation in modern web sites.
Next we'll discuss how our technique for transparently prototyping
applications relates to both developers and users. Lastly we'll look at how
activity feeds fares as a social navigation technique.

\section{Social Navigation on the Social Web}

Trough our study of Flickr and Facebook we've gotten an impression of how
social navigation is used in modern web sites. We'll discuss our most
interesting findings in the next sections.

\subsection{No explicit design for social navigation}

It does not seem like the designers of social web pages design for social
navigation explicitly. There seem to be a trend of designing for social
interaction. As we discussed in
\sectionref{background.sociality.the.social.web.ggg}, the inventor of the Web
now sees social relationships as a highly important part of the modern web.
In addition we've seen in
\sectionref{background.sociality.the.social.web.social.network.sites}
that social network sites have become very popular amongst web citizens.

An implicit by-product of such a design approach seems to be
the creation of several forms of social navigation constructs.
We base this observation on our studies of two large social web sites
(see \chapterref{analysis} for details)
in addition to cursory observations from other social web sites.

\subsection{Social navigation have become mainstream}

We were able to locate navigational mechanisms which could be considered
social navigation in all the web sites we studied. While our view of the
modern web surely is uncomplete, we take this as a sign for an increased use
of social navigation. Social navigation seems to have become mainstream.

The reasons for this increased usage can be many. We think the most dominant
factor is the high focus on creating web sites where social interaction is
supported. As described earlier, social navigation is then implicitly created
when one designs web sites with such a focus.

\subsection{Social navigation advice is given by peers}

An overlying theme of the forms of social navigation we found in the wild were
that it was created by equal peers. The constructs for enabling social
navigation are created by the web site creators. But the data that
enables navigational choices of a social nature are created by other
users\dash{}by the community.

We therefore purpose that navigational advice have to
be given by peers to be considered social navigation,
whether indirect or direct, explicit or implicit.
In other words
navigational advice have to be given by individuals on the same horizontal
level as yourself. This means that navigational advice given by web editors
and web designers\dash{}people vertically superior to yourself\dash{}can not
be considered true social navigation.

\section{Transparent Prototyping with Greasemonkey}

\subsection{Development perspective}

During our development of a prototype application with Greasemonkey for
enhancing an established web page we got a feel for its pros and cons from a
development perspective.

\subsubsection{Requires no access to the established implementation}

\subsubsection{Requires little knowledge of the established implementation}

\subsubsection{Requires more work than altering the established
  implementation}

\subsubsection{Fragile when the established implementation is changed}
% Only happened once during a two month span on \urort{}. What was scary
% thought was that the changes made the user script on the client side
% obsolete. If this had happened under production usage when the user scripts
% was pushed to the clients we would be in a world of trouble. Changes on the
% server side platform can be handled more transparently.

\subsubsection{Less performant than the established implementation}

\subsection{User perspective}

When we conducted a study of our prototype with real world users
we got valuable feedback on how well such a system works for the average user.

\subsubsection{Limited in browser selection}

\subsubsection{Difficulties with installing Greasemonkey}

\subsubsection{Difficulties with installing user-scripts}


\section{Activity Feeds as a Social Navigation Technique}
