\chapter{Conclusion}
\label{chapter:conclusion}

The results of the research which we have provided in this thesis can be
categorized into three venues:

\begin{enum}
  \item We have given a structured overview of the field of social navigation
    as seen both in academic literature and in some noticeable social web
    sites.
  \item We have provided details of how one can implement unobtrusive
    prototypes in established spaces and the feasibility of such a technical
    approach.
  \item We have contributed knowledge of how activity streams functions as
    a social navigation technique on the \urort{} web site.
\end{enum}

Based on these three venues we'll provide the most important lessons to take
away from our research before we discuss possible future work in these fields.

\section{Lessons Learnt}

We believe there are both some theoretical and practical lessons to take away
from our research

\subsection{Social navigation}

As viewed in academic literature social navigation can mean different things.
We proposed a new definition of social navigation based on our belief in the
importance of peers in a social navigation system.%
\sidenote{
  The definition can be found in
  \sectionref{flickr.facebook.discussion.peers}.
}
This means that the information given from other people which guide navigation
have to come from peers within the system where navigation are conducted to be
considered social navigation.

Information given by the creators or 
editors of a web site are therefore not social navigation when used for
navigational purposes.
The creators can however implement structures in their web pages where users
of the system can impose information which can be used for social navigation.
One example of this divide can be found in recommender systems. Content based
recommendations is not social navigation since the information used
in the navigational process are given by the editors of the web page.
Recommendations given by collaborative filtering is on the other hand social
navigation since the navigational information is given by peers in the system.

\subsection{Unobtrusive prototyping}

Creating unobtrusive prototypes with Greasemonkey have its advantages and
disadvantages when used in real world experiments.

Greasemonkey is best fitted for situations where you don't have access to the
established web site one are prototyping on. If one have access to the inner
workings of a web site, it would probably be more efficient and easier to
implement the prototype within the established implementation. Another benefit
of modifying the web site implementation itself is the elimination of the
Greasemonkey and user-script installation process in addition to wider browser
support.

We found the major disadvantage of using Greasemonkey in a real world
experimental setting to be this complicated installation process and limited
browser support. We contribute this as the major factors for the high
non-accomplish rates we witnessed. Having conducted experiments in a
laboratory setting where users used pre-configured machines would have
mediated this problem.

\subsection{Activity streams}

Based on our experiment with activity streams on \urort{} we have inconclusive
findings of the success of such an social navigation technique.

\section{Future Work}

% More widespread collection of social naivation examples from social web
% sites to see if our conclusions holds true. Also a more structured taxonomy
% of social navigation techniques are needed. What we have proivded are
% cursory work in this area and could be expanded on.

% If we had time we would have conducted a laboratory study as well. This way
% we would not have problems with the technological seeding we've seen
% since only the most technical apt were able to complete the entire study.

% We would also have tested the prototypes over a longer time frame. Favorite
% usage numbers gave us inconclusive results. With a longer usage period there
% could be possible changes in how important favorites are.
