\chapter{Introduction}

% The point of the introduction is to answer: what is this thesis about?
% Explain this in four steps by:
%
%   * Why you have chosen this topic rather than any other. Examples:
%       - has been neglected
%       - much discussed but not properly and fully
%   * Why this topic interests you.
%   * The kind of research approach or academic disciple you will utilize.
%   * Your research questions or problems.
%
% The role of the introduction, like the abstract, is to orient your readers.
% This is best done clearly and succinctly.

The web has come a long way since it's inception when it was a global system
for document sharing amongst researchers\citep[p.~82]{bernerslee92}. We've
seen the coming of an increasingly more social web as
``the digital domain has seen a significant growth in the scale and richness
of on-line communities'' \citep[p.~44]{backstrom06}.
\citet[ch.~1, p.~2]{rosa07} reports that the
use of blogs in the general public%
\sidenote{%
  Represented with a total of 6,545 respondents
  to a survey conducted in Canada, France, Germany, Japan,
  the United Kingdom, and the United States.
}
have grown significantly the last 18 months as can be seen in
Table~\ref{table:blog.usage}
(p.~\pageref{table:blog.usage}).
More than one third of people using blogs have contributed actively
by writing blog articles\citep[ch.~1, p.~6]{rosa07}.
It has been argued that web citizens'
familiarity with blogging laid the groundwork for the explosion we are seeing
in user participation in web communities \citep[ch.~2.2]{beer07}.
% maby cite weiss here also

\sidetable{Blog Usage}{%
  \label{table:blog.usage}
  \begin{tabular}{lll}

    \toprule
    Year & Passive & Active \\
    \midrule

    2005 & 16\% & n/a \\
    2007 & 45\% & 17\% \\

  \end{tabular}
}

At the same time advances in hardware and development tools have made it
easier and cheaper to create new web services. We're seeing an
abundance of new companies trying to make it in this field.
During the initial studies of our research we frequented
many of the new offerings in the web services space. Our impression is that
this area of the web infamously coined ``Web 2.0''%
\sidenote{%
  Web 2.0 was first used as the name of a conference arranged by
  O'Reilly Media. The ``2.0'' part of the conference name was then used to
  signify the revival of interest in the web after the dot-com bubble in the
  early 21st century \citep{oreilly07}.
  Later the founder of O'Reilly Media, Tim O'Reilly, defined
  the term as the characteristics of the web services that survived the dot-com
  bubble and the web services he deemed to be the best newcomers to the
  field \citep{oreilly05}.
}
is bringing interesting innovations. As \citet[p.~18]{weiss05} argues:
``When we consider a hot, buzz-worthy Web site of the new Internet evolution
\ldots
they are at the same time incredibly innovative and yet--not''.
User generated content and the notion of \emph{collective intelligence} is
driving the innovation we're seeing in this incrementally new version of the
web. But the underlying principles envisioned by \citet{bush45} as he
described

This thesis focuses on navigational problems and only those which are of a
social nature.
% Navigation is a fundemental principle of the interconnected web (cite).
% If a users can't get a hold of the data they're in need of you've lost
% as a web service provider.
Having efficient and easy to use navigation is clearly
essential for serving your users' interests. As the content on the web to a
higher degree than before are user generated it has become almost impossible
for the creators of a given web service to design sound navigation without
relying on their loyal users. Instead such structures have to be organically
grown by designing navigational schemes that harness the work your user base
is conducting as they use your service. This is the essence of social
navigation as the first definition of the term clearly and succinctly
captures:

\begin{quote}
In social navigation, movement from one item to another is provoked as an
artifact of the activity of another or a group of others. \citep{dourish94} 
\end{quote}

\section{Motivation}

Social navigation are as we've seen a well defined term within the academic
community and there have been a handful of studies on it's applicability in
context of the web \citep{dieberger97,wexelblat99}. % more studies needed
Most of this research took place in the early days of the
web---preceding our current area of Web 2.0.
As \citet{beer07} notes: ``\ldots `internet time' now runs at at a clock speed
several orders of magnitude faster than that of academic research''.
We described earlier the
growth we're seeing of web services with social aspects and
we firmly believe that these are bound to provide
for innovations in the space of navigation. It would therefore be interesting
to look at some of the state-of-the-art social web services and look at what
contributions they have made to the field of social navigation.

We are currently lacking information on how one can use social navigation
consciously in a web application design process. Such navigational schemes
seem to be created without guidance and many times as an afterthought.
It appears that methods for establishing incentives for user participation
is the focal point of web architecture design today. Even though such
approaches can result in sound and interesting navigation it's our impression
that a focus on solving users' navigational problems is more beneficial for
the usability of a web service.

\section{Objective}

By collecting examples of good navigational implementations in the wild,
analyzing them, and categorize them we hope to offer a structured view of the
field of social navigation. We offer this information in the way of a taxonomy
of useful social navigation techniques.
As we are unaware of any established technique for
conducting such a study on real world navigation systems we create our own
method as we go---fine tuning it as we learn from our experiences.

We try to improve an existing web service by using some of the techniques
established in our initial research of how social navigation is leveraged
in real world applications. More specifically we are prototyping possible
navigational improvements for the Norwegian Broadcasting Corporation's joint
TV, radio, and Internet project: \emph{Ur\o{}rt}---a service where artists upload
their demos and get valuable playtime on radio and TV if their products are
judged to be of sufficient quality. Our focus is on the projects
web community\sidenote{Available at \url{http://nrk.no/urort}} where users can
interact in a social manner in addition to uploading their songs.

Going in and making changes to an existing web service can be both an
daunting and time consuming task. First one have to establish a trustworthy
relationship with the creators of such a service so that they are certain
you're not introducing bugs in their production software. Secondly, grasping the
code base, third party libraries, and development tools of such a software
project demands a lot of upfront effort before any real development work can
begin. This goes against the prototypical process we intended to use while
experimenting with Ur\o{}rt.

Even though we've had an ongoing dialog with the developers of Ur\o{}rt we
decided to create our prototype as a layer on top of their service. By using a
extension\sidenote{Specifically Greasemonkey, an extension allowing for
customization of presentation and behavior of web pages. Available at
\url{http://www.greasespot.net}} for the open source Firefox%
\sidenote{Available at \url{http://firefox.com}}
web browser we were able to make changes and additions to the way Ur\o{}rt were
presented to users who were participating in our study. We were able to create
a back end for the additional data and behavior our new functionality required
with the frameworks and programming language we found to be most efficient in
a prototypical process. This resulted in a transparent experience for our end
users as long as they had taken the necessary steps to set up the browser
extension and our script.

With our technical solution in place we were able to test how it performed in
practice by conducting ``some sort of user observation'' (not decided) and
(possibly) more quantitative surveys.

This leads to the research question we had in mind while conducting
the tasks described above:

\begin{quote}
  How can social navigation influence usage of established web services?
\end{quote}

The role of this question have been to give our research direction, show where
it's boundaries were, keep us focused, and point to the needed methods and
data \citep[p.~77]{silverman05}.

% hypotheses ?

\section{Contributions}

Contributions from our research on social navigation is threefold:

\begin{description}
  \item[Informing navigational design] by giving a structured overview of
    various navigational schemes in use today.
  \item[Transparent prototyping techniques] by sharing experiences from
    creating a unobtrusive shell of navigational designs on top of an
    existing web service.
  \item[Applicability of social navigation] by showing results from real
    users' usage of social navigation designs.
\end{description}

\section{Outline}

Moving on from this introductory chapter we'll take a look at \ldots before we
finally conclude our research in the final chapter.
