\chapter{Introduction}
\label{chapter:introduction}

% The point of the introduction is to answer: what is this thesis about?
% Explain this in four steps by:
%
%   * Why you have chosen this topic rather than any other. Examples:
%       - has been neglected
%       - much discussed but not properly and fully
%   * Why this topic interests you.
%   * The kind of research approach or academic disciple you will utilize.
%   * Your research questions or problems.
%
% The role of the introduction, like the abstract, is to orient your readers.
% This is best done clearly and succinctly.

The web has come a long way since its inception when it functioned as a
global interconnected system for document sharing amongst researchers
\citep[\p{82}]{bernerslee92}. We've
seen the coming of an increasingly more social web as
\postquote[\p{44}]{backstrom06}{%
  the digital domain has seen a significant growth in the scale and richness
  of on-line communities}
There have been
an increase from 18\% to 45\%
\begin{sparkline}{3}
  %Name  Size Unit
  %2005  16%  .356
  %2007  45%  1
  \sparkspike .25  .356
  \sparkspike .75  1
\end{sparkline}
in blog usage by the general public%
\sidenote{%
  Represented with a total of 6,545 respondents
  to a survey conducted in Canada, France, Germany, Japan,
  the United Kingdom, and the United States
  by \citet[ch.~1, \p{2}]{rosa07}.
}
in an 18 month period from 2005 to 2007.
It has been argued that web citizens'
familiarity with blogging laid the groundwork for the explosion we are seeing
in user participation in web communities
\citedouble{\p{20}}{weiss05}{\para{2.2}}{beer07}.

At the same time advances in hardware and web development tools have made it
easier and cheaper to create new web sites. We're now seeing an
abundance of new offerings in this field.
It has been argued that many of the concepts this modern web brings are
evolutionary instead of revolutionary \citep[\p{45}]{yakovlev07}.
\citet[\p{17}]{treese06} also witnesses a continuous evolution, but with
exploratory innovations as he notes that most technological changes are
incremental.%
\sidenote[-4]{
  \prequote{knuth07}{%
    also believes innovation in computer science is incremental:}{%
      I firmly believe that computer science advances by thousands of people
      solving small problems, which go together and create a massive edifice.
      Every year that goes by, hardly anything is done that appears to be a
      milestone worthy of mass attention; yet after five or ten years pass,
      the whole field has changed significantly}
}
\begin{fullquote}[\p{18}]{weiss05}{have seen this trend}
  When we consider a hot, buzz-worthy Web site of the new Internet evolution
  [\ldots]
  they are at the same time incredibly innovative and yet\dash{}not.
\end{fullquote}

User generated content and the notion of \term{collective intelligence}%
\sidenote[1]{
  User generated content and collective intelligence is discussed in
  \sectionref{background.sociality.the.social.web}.
}
is one of the major driving forces for the innovation we're seeing in this
incrementally new version of the web. But basically what we're experiencing
today with the World Wide Web and social/collaborative software systems was
envisioned ages ago by \citet{licklider68} and \citet{bush45}.

During the initial studies of our research we frequented
many of the these modern web sites. Our impression is that
this area of the web infamously coined \term{Web 2.0}\dash{}an
increment in version opposed to the age when the Web was in its
infancy\dash{}is bringing interesting innovations. While they might not be
groundbreaking, we justify a closer look at them in this thesis.

\section{Focus}

This thesis have a focus on navigational problems and only those
which are of a social type.%
\sidenote{%
  Take a look at \sectionref{background.social.navigation}
  to learn more about navigational systems with social characteristics.
}
Navigation in context of computer systems is
essentially a metaphor based on how people find their way in the physical
world. So just as a compass and map can be crucial in your ability to find a
cabin deep in the woods during a hike\dash{}reliable and efficient
navigational systems on the web is of uttermost importance when you're trying
to locate a certain electronic object containing valuable information.

In addition to only focusing on \term{social navigation} we're only concerned
which such types of navigation on the Web.
On the Web we're using hyperlinks \citep[\p{90}]{nelson65} to provide users
with navigational choices. We're only focusing on the use of such hyperlinks
within web browsers and not navigation support in auxiliary tools as email
clients, instant messaging clients, and so on. Our focus is further refined by
targeting our research  only on what happens inside various web pages. This
means that other navigation forms supported by the browser itself or third
party extensions or plugins is outside of our scope, as detailed in
\sectionref{background.navigation.navigation.on.the.web}.

While we're aware that search is an important part of peoples every day
navigational behavior we've introduced additional confinements and decided to
only concentrate on browsing behavior (see
\sectionref{background.navigation.navigation.on.the.web} for details).

When studying various web pages it became apparent that some use of
social navigation mechanisms implies pretty large privacy concerns. By mining
users' previous actions specific user profiles can be generated. One can then
represent very sensitive characteristics of individuals such as sexual
orientation, political status, and religious beliefs.
We feel this subject area of social navigation in relation to privacy warrants
a master thesis on its own. Discussion of privacy concerns have therefore
been excluded from our research so that we can look more closely at the
navigational characteristics of social navigation.

\section{Motivation}

\sidetable[Social Navigation in Academia]{%
           Social navigation in academia, by content
           \label{table:social.navigation.academia}}{%
  \begin{tabular}{lr}

    & Articles \\

    \cmidrule(lr){2-2}

    Modern Web & 5 \\
    Other & 21 \\

  \end{tabular}
}

Social navigation are as we'll see in
\sectionref{background.social.navigation}
a well defined term within the academic community.
During our literature review we collected to the best of our abilities all
academic articles where social navigation was discussed. Our approach was to
use keyword search and citation search in the databases listed in
\tablepageref{literature.databases}.%
\sidenote[2]{
  For more about our literature collection method, see
  \sectionref{literature.search}.
}
\tableref{social.navigation.academia} shows the metrics of articles
we found about social navigation in context of the modern web as captured by
the Web 2.0 term (social network sites, folksonomies, and wikis) and other
areas of computer science (classic web, general user interfaces, security, and
so on).
On several occasions we encountered similar articles by the same authors
discussing the same problems and systems. In such circumstances the collection
of two or more similar articles was counted as one.

Our current area of Web 2.0 in relation to navigational problems have in our
view (based on our literature findings) little coverage in academia.
\prequote{beer07}{notes that}{%
 \,`internet time' now runs at at a clock speed several orders
 of magnitude faster than that of academic research}
We described earlier the growth we're seeing of web sites with social
aspects and we believe that some of these provide for novel examples of social
navigation. It would therefore be interesting to look at some of the
state-of-the-art social web sites and look at what contributions they have
made to the field of social navigation.

We are currently lacking information on how one can use social navigation
consciously in a modern web application design process. Such navigational
schemes seem to be created without guidance and many times as an afterthought.
It appears that methods for establishing incentives for user participation
is the focal point of web architecture design today. Even though such
approaches can result in sound and interesting navigation it's our impression
that a focus on solving users' navigational problems is more beneficial for
the usability of a web site. We try to take such an approach when we design a
navigational prototype (see \sectionref{implementation.design} for details).

\section{Objective}

By collecting examples of good navigational implementations in the wild
and analyzing them we hope to give a clearer
view of the field of social navigation in our modern web.
As we are unaware of any established technique for
conducting such a study on real world navigation systems we create our own
method as we go\dash{}fine tuning it as we learn from our experiences.

Using the knowledge we gained from collecting social navigation examples
from real world web sites we try to improve an existing web site by
implementing a navigational prototype. The navigational technique we
decided on implementing is a so called \term{activity stream}. See
\sectionref{analysis.facebook.news.feed} and
\sectionref{implementation.design.activity.stream}
for more information about this particular navigational technique.

The Norwegian Broadcasting Corporation's joint
\abbr{TV}, radio, and internet project \project{\urort{}}\dash{}a site where
artists upload their demos and get valuable playtime on radio and \abbr{TV} if
their products are judged to be of sufficient quality\dash{}was
the candidate for implementing a navigational prototype.
Our focus was on the \urort{} web community%
\sidenote{
  Available at \url{http://nrk.no/urort}.
}
where users can interact in a social manner, listen to other people's songs,
and upload their own creations.
We decided to build our application in an transparent manner on top of the
web site \urort{} offered. The rationale for such a decision can be found in
\sectionref{implementation.building.on.top.of.the.web}.

With our technical solution in place we were able to test how it performed in
practice by conducting an empirical experiment with real world users.

\section{Contributions}

Contributions from our research on social navigation is threefold:

\begin{enum}
  \item Informing navigational design by giving a structured overview of
    various social navigational schemes used in academia and the real world.
  \item Exemplifying transparent prototyping methods by sharing experiences
    with creating an unobtrusive shell of navigational designs on top of an
    existing web site.
  \item Applicability of a particular social navigation technique
    by discussing findings from an experiment of its real world usage.
\end{enum}

\section{Outline}

%%
%% rewrite needed, with bullet points
%%
