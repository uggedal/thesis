\begin{figure}
  \begin{whole}
    \includegraphics{fig_experiment_mortality}
    \reducecaption{10}
    \caption[Experiment Non-Achievement Rates]{
      Non-achievement rates for the various parts of the experiment.
      The sample $N$ of 171 \urort{} users were given a pretest and
      a sample $n_1$ of 123 completes the pretest.
      By randomization a sample $E_1$ of 35 were given a treatment while
      a sample $C_1$ of 36 were given a placebo.
      The sample $n_1$ were given a follow-up questionnaire for checking
      how the installation of the prototypes went. $n_2$ completed the follow-up
      questionnaire.
      A sample $E_2$ of 25 and a sample $C_2$ of 20 managed to install
      the treatment and placebo prototype respectively.
      Of those, a sample $E_3$ of 14 from the treatment sample $E_2$ and
      a sample $C_3$ of 11 from the placebo sample $C_2$ completed
      a posttest.
    }
    \label{figure:fig.experiment.mortality}
  \end{whole}
\end{figure}

\section{Results}
\label{section:empirical.results}

\figureref{fig.experiment.mortality} gives an overview of how many respondents
we had to our pretest, posttest, and follow-up questions. As seen in the
figure the non-achievement rate was quite high from the initial sample $N$ to
the respondents of our pretest $n_1$. The non-achievement
rates was more or less constant from pretest respondents $n_1$ to treatment
$E_1$ and placebo $C_1$ obtention, to successful treatment $E_2$ and placebo
$C_2$ installation, and finally to posttest respondents $E_3$ and $C_3$.
The non-achievement rate for going through
all steps from a pretest ($n = 123$) to the posttest ($n = 14 + 11 = 25$)
\begin{sparkline}{3}
  %Name  Size Unit
  %pre   100%  1
  %post 20.3%  .203
  \sparkspike .25  1
  \sparkspike .75  .203
\end{sparkline}
is quite high at 79.7\%.

\begin{table}
  \begin{whole}
  \begin{tabular}{rrrrrrrrrrl}

    &
    &
    \multicolumn{2}{c}{Gender (\%)} \\

    &
    \multicolumn{1}{c}{$N$} &
    \multicolumn{1}{c}{female} &
    \multicolumn{1}{c}{male} &
    \multicolumn{1}{c}{$\overline{Age}$} &
    \multicolumn{1}{c}{$\sigma$} &
    &
    \multicolumn{1}{c}{$U$} &
    \multicolumn{1}{c}{$Z$} &
    \multicolumn{1}{c}{$p$ (2-tailed)} \\

    \cmidrule(lr){2-10}

    $E$ &
    13 &
    14.3 &
    85.7 &
    26.23 &
    10.902 &
    \multirow{3}{*}{\threeguides} &
    68.5 &
    -0.174 &
    0.876 &
    $E$\dash{}$C$ \\

    $C$ &
    11 &
    --- &
    100.0 &
    24.55 &
    7.917 &
    &
    580.0 &
    -0.468 &
    0.645 &
    $E$\dash{}Rest \\

    Rest &
    97 &
    8.2 &
    91.8 &
    26.81 &
    9.324 &
    &
    475.0 &
    -0.595 &
    0.559 &
    $C$\dash{}Rest \\

  \end{tabular}
  \reducecaption{3}
  \caption[Respondents Gender and Age, Between Groups]{%
    Respondents gender and age. Comparison of age between
    experiment group, control group, and those not given treatment
    or placebo.
  }
  \label{table:respondents.profile.gender.age}
  \end{whole}
\end{table}

\tableref{respondents.profile.gender.age} shows the gender distribution and
mean age of both our experiment group, control group, and the respondents
which never got a treatment or placebo. The age differences between groups are
unnoticeable and far from statistically significant.
The data shows that the majority of respondents were male.

In addition to gather characteristics about age and gender we also
investigated how frequent respondents used Firefox as their browser,
how often they used \urort{}, and how often they used \urort{} in an
authenticated state.
Based on the questions:
\begin{items}
  \item Do you use Firefox{}?
  \item Do you sign-in (with user name and password) when using
    \urort{}?
\end{items}

we graded respondents answers as follows:

\begin{items}
  \item Always: 5
  \item Regularly: 4
  \item Sometimes: 3
  \item Seldom/never: 2
\end{items}

Based on the question:

\begin{items}
  \item How often do you use \urort{}?
\end{items}

we graded respondents answers as follows:

\begin{items}
  \item Daily: 5
  \item Several times a week: 4
  \item Weekly: 3
  \item Monthly: 2
  \item Seldom/never: 1
\end{items}

\begin{table}
  \begin{whole}
  \begin{tabular}{rrrrrrrrll}

    &
    \multicolumn{1}{c}{$N$} &
    \multicolumn{1}{c}{$Mdn$} &
    \multicolumn{1}{c}{$Rng$} &
    &
    \multicolumn{1}{c}{$U$} &
    \multicolumn{1}{c}{$Z$} &
    \multicolumn{1}{c}{$p$ (2-tailed)} \\

    \cmidrule(lr){2-8}

    $E$ &
    14 &
    5 &
    3 &
    \multirow{3}{*}{\threeguides} &
    70.5 &
    -0.407 &
    0.707 &
    $E$\dash{}$C$ &
    \multirow{3}{*}{Firefox usage} \\

    $C$ &
    11 &
    5 &
    2 &
    &
    668.0 &
    -0.181 &
    0.887 &
    $E$\dash{}Rest \\

    Rest &
    98 &
    5 &
    3 &
    &
    508.5 &
    -0.352 &
    0.748 &
    $C$\dash{}Rest \\

    \cmidrule(lr){2-9}

    $E$ &
    14 &
    3 &
    4 &
    \multirow{3}{*}{\threeguides} &
    47.5 &
    -1.757 &
    0.085 &
    $E$\dash{}$C$ &
    \multirow{3}{*}{\urort{} usage} \\

    $C$ &
    11 &
    4 &
    2 &
    &
    653.5 &
    -0.294 &
    0.773 &
    $E$\dash{}Rest \\

    Rest &
    98 &
    3 &
    4 &
    &
    353.5 &
    -1.918 &
    0.055 &
    $C$\dash{}Rest \\

    \cmidrule(lr){2-9}

    $E$ &
    14 &
    4 &
    3 &
    \multirow{3}{*}{\threeguides} &
    63.5 &
    -0.774 &
    0.462 &
    $E$\dash{}$C$ &
    \multirow{3}{*}{Authenticated usage} \\

    $C$ &
    11 &
    3 &
    3 &
    &
    572.5 &
    -1.035 &
    0.306 &
    $E$\dash{}Rest \\

    Rest &
    98 &
    3 &
    3 &
    &
    527.5 &
    -0.125 &
    0.920 &
    $C$\dash{}Rest \\

  \end{tabular}
  \reducecaption{3}
  \caption[Respondents Firefox and \urort{} Usage, Between Groups]{%
    Respondents Firefox and \urort{} usage. Comparison between
    experiment group, control group, and those not given treatment
    or placebo.
  }
  \label{table:respondents.profile.usage}
  \end{whole}
\end{table}

The data shows little difference in Firefox usage between
groups\dash{}the differences are
far from statistically significant. The same trend can be observed for
whether the respondents are authenticated when they use \urort{}.

In actual \urort{} usage, we observe that control group respondents had
higher usage frequencies, both compared to experiment respondents and
respondents which never were given a treatment or placebo. The difference is
close to statistically significant, compared between the control group and
those without treatment or placebo.

\subsection{%
  \rp[1]{
    Do users perceive social navigation through activity streams as helpful in
    order to keep up-to-date on favorites' activities on \urort{}?
  }
}

Our $H_0$ stated that we would not see any positive change in how easy
respondents felt it was to keep up-to-date on favorite's activities
after introducing an activity steam. Our $H_A$ said that we
would indeed observe a change in how respondents rated this task.

\subsubsection{Activities in general}

Based on the statement
\q{It's easy to keep up-to-date on what my favorites are doing
on \urort{}} we graded respondents answers as follows: 

\begin{items}
  \item Fully disagree: 1
  \item Somewhat disagree: 2
  \item Neither agree nor disagree: 3
  \item Somewhat agree: 4
  \item Fully agree: 5
\end{items}

First we compared how easy respondents felt it was to keep up-to-date on
favorites' activities after they were given a treatment (an activity stream)
or a placebo.
See
\tableref{up.to.date.favorite.activities.between} for the results.

\begin{table}
  \begin{tabular}{rrrclrrrr}

    &
    &
    &
    \multicolumn{2}{c}{$Mdn_{\sum{E}}$} \\

    &
    \multicolumn{1}{c}{$N$} &
    \multicolumn{1}{c}{$Mdn$} &
    \multicolumn{2}{c}{$- Mdn_{\sum{C}}$} &
    \multicolumn{1}{c}{$Rng$} &
    \multicolumn{1}{c}{$U$} &
    \multicolumn{1}{c}{$Z$} &
    \multicolumn{1}{c}{$p$ (1-tailed)} \\

    \cmidrule(lr){2-9}

    $E$ &
    12 &
    5 &
    \multirow{2}{*}{\twoguides} &
    \multirow{2}{*}{1} &
    4 &
    \multirow{2}{*}{45} &
    \multirow{2}{*}{-1.353} &
    \multirow{2}{*}{0.089}\\

    $C$ &
    11 &
    4 &
    &
    &
    3 \\

  \end{tabular}
  \caption[Up-to-date on Favorites' Activities, Between Groups]{%
    Up-to-date on favorites' activities. Comparison between
    experiment and control group for the posttest.
  }
  \label{table:up.to.date.favorite.activities.between}
\end{table}

The results show a tendency towards higher acceptance scores for experiment
respondents having used an activity stream, than
control respondents with a placebo. This difference is however
not statistically significant.

Then we tested if there was a change in acceptance within the respondent
groups from before they were given a treatment or placebo and after.
The results can be seen in
\tableref{up.to.date.favorite.activities.within}.

\begin{table}
  \begin{whole}
  \begin{tabular}{rrrrccclrrrr}

    &
    &
    &
    &
    \multicolumn{2}{c}{Post} &
    \multicolumn{2}{c}{$Mdn_{\sum{E}}$} \\

    &
    &
    \multicolumn{1}{c}{$N$} &
    \multicolumn{1}{c}{$Mdn$} &
    \multicolumn{2}{c}{$-$ Pre} &
    \multicolumn{2}{c}{$- Mdn_{\sum{C}}$} &
    \multicolumn{1}{c}{$Rng$} &
    \multicolumn{1}{c}{$T$} &
    \multicolumn{1}{c}{$Z$} &
    \multicolumn{1}{c}{$p$ (1-tailed)} \\

    \cmidrule(lr){3-12}

    \multirow{2}{*}{$E$} &
    Pre &
    14 &
    3 &
    \multirow{2}{*}{\twoguides} &
    \multirow{2}{*}{2} &
    \multirow{4}{*}{\fourguides} &
    \multirow{4}{*}{1} &
    4 &
    \multirow{2}{*}{10} &
    \multirow{2}{*}{-1.513} &
    \multirow{2}{*}{0.086}\\

    &
    Post &
    12 &
    5 &
    &
    &
    &
    &
    4 \\

    \multirow{2}{*}{$C$} &
    Pre &
    11 &
    3 &
    \multirow{2}{*}{\twoguides} &
    \multirow{2}{*}{1} &
    &
    &
    3 &
    \multirow{2}{*}{16} &
    \multirow{2}{*}{-0.787} &
    \multirow{2}{*}{0.258}\\

    &
    Post &
    11 &
    4 &
    &
    &
    &
    &
    3 \\

  \end{tabular}
  \reducecaption{9}
  \caption[Up-to-date on Favorites' Activities, Within Groups]{%
    Up-to-date on favorites' activities.  Comparison between
    pretest and posttest within the experiment and control group.
  }
  \label{table:up.to.date.favorite.activities.within}
  \end{whole}
\end{table}

The data shows a tendency towards higher acceptance rates from the pretest to
the posttest for both experiment and control respondents. The acceptance of
the statement about how easy it was to keep up-to-date on favorites'
activities have risen more for the respondents having used an activity stream
than those with a placebo. Neither of these increases
are statistically significant.

\subsubsection{Specific activities}

We'll now look at respondents answers to more specific statements about
keeping up-to-date on activities.
Based on the statements:
\begin{items}
  \item It's easy to keep up-to-date on whether my favorites publishes
    new songs on \urort{}.
  \item It's easy to keep up-to-date on whether my favorites publishes
    new blog posts on \urort{}.
  \item It's easy to keep up-to-date on whether my favorites are
    performing at concerts.
  \item It's easy to keep up-to-date on the reactions other users at
    \urort{} have towards my favorite artists' songs.
\end{items}

we graded respondents answers as follows:

\begin{items}
  \item Fully disagree: 1
  \item Somewhat disagree: 2
  \item Neither agree nor disagree: 3
  \item Somewhat agree: 4
  \item Fully agree: 5
\end{items}

We compared how easy respondents felt it was to keep up-to-date on
favorites' specific activities after they are given a treatment
or a placebo.
See
\tableref{up.to.date.favorite.specific.activities.between} for the results.

\begin{table}
  \begin{whole}
  \begin{tabular}{rrrclrrrrl}

    &
    &
    &
    \multicolumn{2}{c}{$Mdn_{\sum{E}}$} \\

    &
    \multicolumn{1}{c}{$N$} &
    \multicolumn{1}{c}{$Mdn$} &
    \multicolumn{2}{c}{$- Mdn_{\sum{C}}$} &
    \multicolumn{1}{c}{$Rng$} &
    \multicolumn{1}{c}{$U$} &
    \multicolumn{1}{c}{$Z$} &
    \multicolumn{1}{c}{$p$ (1-tailed)} \\

    \cmidrule(lr){2-9}

    $E$ &
    12 &
    5 &
    \multirow{2}{*}{\twoguides} &
    \multirow{2}{*}{1,0} &
    4 &
    \multirow{2}{*}{59} &
    \multirow{2}{*}{-0.479} &
    \multirow{2}{*}{0.340} &
    \multirow{2}{*}{Song}\\

    $C$ &
    11 &
    4 &
    &
    &
    2 \\

    \cmidrule(lr){2-9}

    $E$ &
    12 &
    4.5 &
    \multirow{2}{*}{\twoguides} &
    \multirow{2}{*}{1.5} &
    4 &
    \multirow{2}{*}{59} &
    \multirow{2}{*}{-0.460} &
    \multirow{2}{*}{0.289} &
    \multirow{2}{*}{Blog}\\

    $C$ &
    11 &
    3 &
    &
    &
    2 \\

    \cmidrule(lr){2-9}

    $E$ &
    12 &
    3.5 &
    \multirow{2}{*}{\twoguides} &
    \multirow{2}{*}{-0.5} &
    4 &
    \multirow{2}{*}{66} &
    \multirow{2}{*}{0.000} &
    \multirow{2}{*}{0.517} &
    \multirow{2}{*}{Concert}\\

    $C$ &
    11 &
    4 &
    &
    &
    3 \\

    \cmidrule(lr){2-9}

    $E$ &
    12 &
    3.5 &
    \multirow{2}{*}{\twoguides} &
    \multirow{2}{*}{0.5} &
    4 &
    \multirow{2}{*}{60} &
    \multirow{2}{*}{-0.385} &
    \multirow{2}{*}{0.372} &
    \multirow{2}{*}{Review}\\

    $C$ &
    11 &
    3 &
    &
    &
    3 \\

  \end{tabular}
  \reducecaption{11}
  \caption[Up-to-date on Specific Activities, Between Groups]{%
    Up-to-date on favorites' specific activities. Comparison between
    experiment and control group for the posttest.
  }
  \label{table:up.to.date.favorite.specific.activities.between}
  \end{whole}
\end{table}

The reported degree of how easy it is to keep up-to-date
on new songs, blogs posts, and reviews is higher for those
which used an activity stream than those without. Respondents using
and activity stream reported a lower values of how easy it was
to keep up-to-date on concerts compared to those without such treatment.
Neither the tendencies towards increases and decreases in degrees of keeping
up-to-date, from the
control group to the experiment group, were statically significant.

As for general activities, we also tested whether respondents perception
of how easy it was to keep up-to-date on specific activities from
favorites had changed from before a treatment or placebo was given, to after.
See 
\tableref{up.to.date.favorite.specific.activities.within} for the results of the
in group comparisons.

\begin{table}
  \begin{whole}
  \begin{tabular}{rrrrccclrrrrl}

    &
    &
    &
    &
    \multicolumn{2}{c}{Post} &
    \multicolumn{2}{c}{$Mdn_{\sum{E}}$} \\

    &
    &
    \multicolumn{1}{c}{$N$} &
    \multicolumn{1}{c}{$Mdn$} &
    \multicolumn{2}{c}{$-$ Pre} &
    \multicolumn{2}{c}{$- Mdn_{\sum{C}}$} &
    \multicolumn{1}{c}{$Rng$} &
    \multicolumn{1}{c}{$T$} &
    \multicolumn{1}{c}{$Z$} &
    \multicolumn{1}{c}{$p$ (1-tailed)} \\

    \cmidrule(lr){3-12}

      \multirow{2}{*}{$E$} &
        Pre &
        14 &
        4 &
        \multirow{2}{*}{\twoguides} &
        \multirow{2}{*}{1} &
        \multirow{4}{*}{\fourguides} &
        \multirow{4}{*}{0} &
        4 &
        \multirow{2}{*}{8.0} &
        \multirow{2}{*}{-1.780} &
        \multirow{2}{*}{0.049} &
        \multirow{4}{*}{Song} \\

        &
        Post &
        12 &
        5 &
        &
        &
        &
        &
        4 \\

      \multirow{2}{*}{$C$} &
        Pre &
        11 &
        3 &
        \multirow{2}{*}{\twoguides} &
        \multirow{2}{*}{1} &
        &
        &
        3 &
        \multirow{2}{*}{4.0} &
        \multirow{2}{*}{-1.709} &
        \multirow{2}{*}{0.055} &
        \\

        &
        Post &
        11 &
        4 &
        &
        &
        &
        &
        2 \\

    \cmidrule(lr){2-12}

      \multirow{2}{*}{$E$} &
        Pre &
        14 &
        3 &
        \multirow{2}{*}{\twoguides} &
        \multirow{2}{*}{1.5} &
        \multirow{4}{*}{\fourguides} &
        \multirow{4}{*}{1.5} &
        3 &
        \multirow{2}{*}{9.5} &
        \multirow{2}{*}{-1.872} &
        \multirow{2}{*}{0.039} &
        \multirow{4}{*}{Blog} \\

        &
        Post &
        12 &
        4.5 &
        &
        &
        &
        &
        4 \\

      \multirow{2}{*}{$C$} &
        Pre &
        11 &
        3 &
        \multirow{2}{*}{\twoguides} &
        \multirow{2}{*}{0} &
        &
        &
        3 &
        \multirow{2}{*}{13.0} &
        \multirow{2}{*}{-1.150} &
        \multirow{2}{*}{0.166} &
        \\

        &
        Post &
        11 &
        3 &
        &
        &
        &
        &
        2 \\

    \cmidrule(lr){2-12}

      \multirow{2}{*}{$E$} &
        Pre &
        14 &
        3 &
        \multirow{2}{*}{\twoguides} &
        \multirow{2}{*}{0.5} &
        \multirow{4}{*}{\fourguides} &
        \multirow{4}{*}{-0.5} &
        3 &
        \multirow{2}{*}{7.5} &
        \multirow{2}{*}{-1.127} &
        \multirow{2}{*}{0.188} &
        \multirow{4}{*}{Concert} \\

        &
        Post &
        12 &
        3.5 &
        &
        &
        &
        &
        4 \\

      \multirow{2}{*}{$C$} &
        Pre &
        11 &
        3 &
        \multirow{2}{*}{\twoguides} &
        \multirow{2}{*}{1} &
        &
        &
        3 &
        \multirow{2}{*}{5.5} &
        \multirow{2}{*}{-1.063} &
        \multirow{2}{*}{0.203} &
        \\

        &
        Post &
        11 &
        4 &
        &
        &
        &
        &
        3 \\

    \cmidrule(lr){2-12}

      \multirow{2}{*}{$E$} &
        Pre &
        13 &
        3 &
        \multirow{2}{*}{\twoguides} &
        \multirow{2}{*}{0.5} &
        \multirow{4}{*}{\fourguides} &
        \multirow{4}{*}{1.5} &
        4 &
        \multirow{2}{*}{4.5} &
        \multirow{2}{*}{-1.298} &
        \multirow{2}{*}{0.125} &
        \multirow{4}{*}{Review} \\

        &
        Post &
        12 &
        3.5 &
        &
        &
        &
        &
        4 \\

      \multirow{2}{*}{$C$} &
        Pre &
        11 &
        4 &
        \multirow{2}{*}{\twoguides} &
        \multirow{2}{*}{-1} &
        &
        &
        3 &
        \multirow{2}{*}{9.5} &
        \multirow{2}{*}{-0.780} &
        \multirow{2}{*}{0.281} &
        \\

        &
        Post &
        11 &
        3 &
        &
        &
        &
        &
        3 \\

  \end{tabular}
  \reducecaption{3}
  \caption[Up-to-date on Specific Activities, Within Groups]{%
    Up-to-date on favorites' specific activities. Comparison between
    pretest and posttest within the experiment and control group.
  }
  \label{table:up.to.date.favorite.specific.activities.within}
  \end{whole}
\end{table}

The within group data shows that experiment respondents and control
respondents find it easier to keep up-to-date on recent songs
and concert performances after the treatment or placebo was given.
The increases are of the same order for both those which used an
activity stream and those without. Neither of the two groups increases
for concerts is statistically significant.
The experiment groups' increase is significant for songs while the control
groups' increase for songs is insignificant.
Note that the difference between $p$ values for the two groups related to
songs are marginal.

Respondents having used an activity stream saw an increase in median
response from \q{neither agree or disagree} to \q{fully agree} regarding
how easy it was to keep up-to-date on blog posts of favorites.
This difference is statistically significant.
The control group saw no such increase after they were given a placebo.

On the topic of how easy respondents could keep up-to-date on
reviews of favorites' songs the experiment group saw no change
after treatment while the control group saw a decrease after having used
a placebo. This decrease is not statistically significant.

\subsubsection{Perceived usefulness}

Like \citet{davis89} we asked respondents about the perceived usefulness
of the prototype they were given (an activity stream or a placebo) in relation
to keeping up-to-date on favorites. Based on the statements:

\begin{items}
  \item \latest{} would enable me to keep up-to-date on my favorites in an
    efficient manner.
  \item \latest{} would enable me to keep up-to-date on more favorites.
  \item \latest{} would make it easier to keep up-to-date on favorites.
  \item \latest{} would be useful for keeping up-to-date on favorites.
\end{items}

were graded respondents answers as follows: 

\begin{items}
  \item Extremely unlikely: 1
  \item Unlikely: 2
  \item Slight unlikely: 3
  \item Neutral: 4
  \item Slight likely: 5
  \item Likely: 6
  \item Extremely Likely: 7
\end{items}

As these statements were only given in the
posttest we compared how efficient, quantifiable, easy, and useful respondents
felt it was to
keep up-to-date on favorites' specific activities after they had used
an activity stream or a placebo. In other words we compared the difference
between the experiment and control group.
See
\tableref{up.to.date.favorite.perceived.usefulness.between} for the results.

\begin{table}
  \begin{whole}
  \begin{tabular}{rrrclrrrrl}

    &
    &
    &
    \multicolumn{2}{c}{$Mdn_{\sum{E}}$} \\

    &
    \multicolumn{1}{c}{$N$} &
    \multicolumn{1}{c}{$Mdn$} &
    \multicolumn{2}{c}{$- Mdn_{\sum{C}}$} &
    \multicolumn{1}{c}{$Rng$} &
    \multicolumn{1}{c}{$U$} &
    \multicolumn{1}{c}{$Z$} &
    \multicolumn{1}{c}{$p$ (1-tailed)} \\

    \cmidrule(lr){2-9}

    $E$ &
    12 &
    6.5 &
    \multirow{2}{*}{\twoguides} &
    \multirow{2}{*}{0.5} &
    3 &
    \multirow{2}{*}{58.0} &
    \multirow{2}{*}{-0.547} &
    \multirow{2}{*}{0.355} &
    \multirow{2}{*}{Effective}\\

    $C$ &
    11 &
    6 &
    &
    &
    2 \\

    \cmidrule(lr){2-9}

    $E$ &
    12 &
    6 &
    \multirow{2}{*}{\twoguides} &
    \multirow{2}{*}{0} &
    3 &
    \multirow{2}{*}{62.0} &
    \multirow{2}{*}{-0.270} &
    \multirow{2}{*}{0.387} &
    \multirow{2}{*}{More}\\

    $C$ &
    11 &
    6 &
    &
    &
    2 \\

    \cmidrule(lr){2-9}

    $E$ &
    12 &
    6.5 &
    \multirow{2}{*}{\twoguides} &
    \multirow{2}{*}{0.5} &
    3 &
    \multirow{2}{*}{55.5} &
    \multirow{2}{*}{-0.707} &
    \multirow{2}{*}{0.286} &
    \multirow{2}{*}{Easier}\\

    $C$ &
    11 &
    6 &
    &
    &
    2 \\

    \cmidrule(lr){2-9}

    $E$ &
    12 &
    6 &
    \multirow{2}{*}{\twoguides} &
    \multirow{2}{*}{0} &
    3 &
    \multirow{2}{*}{60.5} &
    \multirow{2}{*}{-0.370} &
    \multirow{2}{*}{0.400} &
    \multirow{2}{*}{Useful}\\

    $C$ &
    11 &
    6 &
    &
    &
    2 \\

  \end{tabular}
  \reducecaption{10}
  \caption[Perceived Usefulness, Between Groups]{%
    Perceived usefulness of \latest{}. Comparison between
    experiment and control group for the posttest.
  }
  \label{table:up.to.date.favorite.perceived.usefulness.between}
  \end{whole}
\end{table}

The data shows no differences in ranking regarding \latest{}
perceived ability for enabling respondents to keep up-to-date on more
favorites and general usefulness for keeping up-to-date on favorites,
between the experiment and control group.
The experiment group shows a small increase in the ranking of
\latest{} effectiveness and easefulness for keeping up-to-date on favorites
compared to the control group. These differences is not statistically
significant.

\subsubsection{Perceived ease of use}

Borrowing from the works of \citet{davis89} again,
we asked respondents about the perceived ease of use
of the prototype they were given (an activity stream or a placebo).
Based on the statements:
\begin{items}
  \item It would be easy to learn to use \latest{}.
  \item It would be easy to make \latest{} do what I want.
  \item It would be easy to become skillful at using \latest{}.
  \item \latest{} would be easy to use.
\end{items}

were graded respondents answers as follows:

\begin{items}
  \item Extremely unlikely: 1
  \item Unlikely: 2
  \item Slight unlikely: 3
  \item Neutral: 4
  \item Slight likely: 5
  \item Likely: 6
  \item Extremely Likely: 7
\end{items}

Just as the perceived usefulness statements, these statements were only
given in the posttest. We compared the responses to these statements
between the experiment and control group to see if the introduction
of an activity stream or placebo made a difference in how easy
the respondents perceived \latest{} would be to use.
See
\tableref{up.to.date.favorite.perceived.ease.of.use.between} for the results.

\begin{table}
  \begin{whole}
  \begin{tabular}{rrrclrrrrl}

    &
    &
    &
    \multicolumn{2}{c}{$Mdn_{\sum{E}}$} \\

    &
    \multicolumn{1}{c}{$N$} &
    \multicolumn{1}{c}{$Mdn$} &
    \multicolumn{2}{c}{$- Mdn_{\sum{C}}$} &
    \multicolumn{1}{c}{$Rng$} &
    \multicolumn{1}{c}{$U$} &
    \multicolumn{1}{c}{$Z$} &
    \multicolumn{1}{c}{$p$ (1-tailed)} \\

    \cmidrule(lr){2-9}

    $E$ &
    12 &
    6 &
    \multirow{2}{*}{\twoguides} &
    \multirow{2}{*}{0} &
    3 &
    \multirow{2}{*}{57.5} &
    \multirow{2}{*}{-0.561} &
    \multirow{2}{*}{0.320} &
    \multirow{2}{*}{Easy to learn}\\

    $C$ &
    11 &
    6 &
    &
    &
    3 \\

    \cmidrule(lr){2-9}

    $E$ &
    12 &
    5.5 &
    \multirow{2}{*}{\twoguides} &
    \multirow{2}{*}{-0.5} &
    3 &
    \multirow{2}{*}{50.5} &
    \multirow{2}{*}{-1.025} &
    \multirow{2}{*}{0.177} &
    \multirow{2}{*}{Flexible}\\

    $C$ &
    11 &
    6 &
    &
    &
    4 \\

    \cmidrule(lr){2-9}

    $E$ &
    12 &
    6 &
    \multirow{2}{*}{\twoguides} &
    \multirow{2}{*}{0} &
    3 &
    \multirow{2}{*}{58.5} &
    \multirow{2}{*}{-0.486} &
    \multirow{2}{*}{0.343} &
    \multirow{2}{*}{Become skillful}\\

    $C$ &
    11 &
    6 &
    &
    &
    3 \\

    \cmidrule(lr){2-9}

    $E$ &
    12 &
    6 &
    \multirow{2}{*}{\twoguides} &
    \multirow{2}{*}{0} &
    4 &
    \multirow{2}{*}{59.5} &
    \multirow{2}{*}{-0.429} &
    \multirow{2}{*}{0.372} &
    \multirow{2}{*}{Easy to use}\\

    $C$ &
    11 &
    6 &
    &
    &
    3 \\

  \end{tabular}
  \reducecaption{7}
  \caption[Perceived Ease of Use, Between Groups]{%
    Perceived ease of use for \latest{}. Comparison between
    experiment and control group for the posttest.
  }
  \label{table:up.to.date.favorite.perceived.ease.of.use.between}
  \end{whole}
\end{table}

The ease of use data shows little differences between the groups. The
respondents perception of how flexible \latest{} would be in use are
actually higher for the control group\dash{}does not using an activity stream
but a placebo. This difference is not statistically significant.

\subsubsection{Activity stream as standard feature}

We wanted to measure how well percieved the prototype was,
by asking if people would want the functionality we provided to be a standard
feature on the \urort{} web site. Based on the question:

\begin{items}
  \item Do you think \latest{} should be a standard feature of \urort{}?
\end{items}

we graded respondents answers as follows:

\begin{items}
  \item Fully disagree: 1
  \item Somewhat disagree: 2
  \item Neither agree nor disagree: 3
  \item Somewhat agree: 4
  \item Fully agree: 5
\end{items}

This question was naturally only asked in the posttest after usage.
\tableref{up.to.date.standard.feature.between} compares responses between the
experiment and control group.

\begin{table}
  \begin{tabular}{rrrclrrrr}

    &
    &
    &
    \multicolumn{2}{c}{$Mdn_{\sum{E}}$} \\

    &
    \multicolumn{1}{c}{$N$} &
    \multicolumn{1}{c}{$Mdn$} &
    \multicolumn{2}{c}{$- Mdn_{\sum{C}}$} &
    \multicolumn{1}{c}{$Rng$} &
    \multicolumn{1}{c}{$U$} &
    \multicolumn{1}{c}{$Z$} &
    \multicolumn{1}{c}{$p$ (1-tailed)} \\

    \cmidrule(lr){2-9}

    $E$ &
    11 &
    5 &
    \multirow{2}{*}{\twoguides} &
    \multirow{2}{*}{0} &
    1 &
    \multirow{2}{*}{55.0} &
    \multirow{2}{*}{-0.607} &
    \multirow{2}{*}{0.500}\\

    $C$ &
    11 &
    5 &
    &
    &
    1 \\

  \end{tabular}
  \caption[The Prototype as a Standard Feature,
           Between Groups]{%
    The prototype as a standard feature. Comparison
    between experiment and control group for the posttest.
  }
  \label{table:up.to.date.standard.feature.between}
\end{table}

The standard feature data shows equally positive results for both groups.
There is no significant difference in the results.

\subsection{%
  \rp[2]{
    Does social navigation through activity streams lead users to more often
    keep up-to-date on favorites' activities on \urort{}?
  }
}

Our $H_0$ stated that we would not see any increase in how frequent
experiment respondents kept up-to-date on favorites' activities after giving
them an activity stream. The alternative $H_A$ contradicted this and said
an activity stream would increase the frequency of how often experiment
respondents were keeping up-to-date.

\subsubsection{Keeping up-to-date frequency}

Based on the question
\q{How often do you update yourself on what your favorites on \urort{}
   are doing?}
we graded respondents answers as follows: 

\begin{items}
  \item Daily: 5
  \item Several times a week: 4
  \item Weekly: 3
  \item Monthly: 2
  \item Seldom/never: 1
\end{items}

First we compared how frequent respondents having used an activity stream kept
up-to-date with those which hadn't used an activity stream. The results are
displayed in \tableref{up.to.date.favorite.activities.frequency.between}.

\begin{table}
  \begin{tabular}{rrrclrrrr}

    &
    &
    &
    \multicolumn{2}{c}{$Mdn_{\sum{E}}$} \\

    &
    \multicolumn{1}{c}{$N$} &
    \multicolumn{1}{c}{$Mdn$} &
    \multicolumn{2}{c}{$- Mdn_{\sum{C}}$} &
    \multicolumn{1}{c}{$Rng$} &
    \multicolumn{1}{c}{$U$} &
    \multicolumn{1}{c}{$Z$} &
    \multicolumn{1}{c}{$p$ (1-tailed)} \\

    \cmidrule(lr){2-9}

    $E$ &
    13 &
    2 &
    \multirow{2}{*}{\twoguides} &
    \multirow{2}{*}{-1} &
    2 &
    \multirow{2}{*}{24.5} &
    \multirow{2}{*}{-2.955} &
    \multirow{2}{*}{0.002}\\

    $C$ &
    11 &
    3 &
    &
    &
    2 \\

  \end{tabular}
  \caption[Up-to-date on Activities Frequency,
           Between Groups]{%
    Frequency of keeping up-to-date on favorites' activities. Comparison
    between experiment and control group for the posttest.
  }
  \label{table:up.to.date.favorite.activities.frequency.between}
\end{table}

As can be seen from the data, respondents which had used an activity stream
reported lower frequencies of use than those which had used a placebo.
This difference is highly significant.

As for our previous two research questions and hypotheses we've also checked
if there were changes in the frequency of keeping up-to-date within the
experiment and control groups from before they were given a treatment or
placebo, to after.
The results can be seen in
\tableref{up.to.date.favorite.activities.frequency.within}.

\begin{table}
  \begin{whole}
  \begin{tabular}{rrrrccclrrrr}

    &
    &
    &
    &
    \multicolumn{2}{c}{Post} &
    \multicolumn{2}{c}{$Mdn_{\sum{E}}$} \\

    &
    &
    \multicolumn{1}{c}{$N$} &
    \multicolumn{1}{c}{$Mdn$} &
    \multicolumn{2}{c}{$-$ Pre} &
    \multicolumn{2}{c}{$- Mdn_{\sum{C}}$} &
    \multicolumn{1}{c}{$Rng$} &
    \multicolumn{1}{c}{$T$} &
    \multicolumn{1}{c}{$Z$} &
    \multicolumn{1}{c}{$p$ (1-tailed)} \\

    \cmidrule(lr){3-12}

    \multirow{2}{*}{$E$} &
    Pre &
    13 &
    1 &
    \multirow{2}{*}{\twoguides} &
    \multirow{2}{*}{1} &
    \multirow{4}{*}{\fourguides} &
    \multirow{4}{*}{1} &
    2 &
    \multirow{2}{*}{10} &
    \multirow{2}{*}{-1.941} &
    \multirow{2}{*}{0.046}\\

    &
    Post &
    13 &
    2 &
    &
    &
    &
    &
    2 \\

    \multirow{2}{*}{$C$} &
    Pre &
    11 &
    3 &
    \multirow{2}{*}{\twoguides} &
    \multirow{2}{*}{0} &
    &
    &
    2 &
    \multirow{2}{*}{8} &
    \multirow{2}{*}{-1.134} &
    \multirow{2}{*}{0.227}\\

    &
    Post &
    11 &
    3 &
    &
    &
    &
    &
    2 \\

  \end{tabular}
  \reducecaption{9}
  \caption[Up-to-date on Activities Frequency, Within Groups]{%
    Frequency of keeping up-to-date on favorites' activities. Comparison
    between pretest and posttest within the experiment and control group.
  }
  \label{table:up.to.date.favorite.activities.frequency.within}
  \end{whole}
\end{table}

The results show a different picture than the between group posttest results.
Here we see that the frequency of keeping up-to-date have increased over time
for those respondents which used an activity stream. Those respondents with a
placebo show a stagnation in the frequency of keeping up-to-date.
The increase for experiment respondents is statistically significant.

\subsubsection{Prototype usage frequency}

We did also collect information in the posttest of how frequent respondents
had used the prototype. Based on the question:

\begin{items}
  \item How frequently have you used \latest{} when you are
    signed-in on \urort{}?
\end{items}

we rated respondents answers as follows:

\begin{items}
  \item Have not used: 1
  \item Only a few times: 2
  \item almost every time: 3
  \item every time: 4
\end{items}

\tableref{up.to.date.prototype.frequency.between} lists an comparison
between the experiment and control group for this question and ratings.

\begin{table}
  \begin{tabular}{rrrclrrrr}

    &
    &
    &
    \multicolumn{2}{c}{$Mdn_{\sum{E}}$} \\

    &
    \multicolumn{1}{c}{$N$} &
    \multicolumn{1}{c}{$Mdn$} &
    \multicolumn{2}{c}{$- Mdn_{\sum{C}}$} &
    \multicolumn{1}{c}{$Rng$} &
    \multicolumn{1}{c}{$U$} &
    \multicolumn{1}{c}{$Z$} &
    \multicolumn{1}{c}{$p$ (1-tailed)} \\

    \cmidrule(lr){2-9}

    $E$ &
    12 &
    3 &
    \multirow{2}{*}{\twoguides} &
    \multirow{2}{*}{1} &
    2 &
    \multirow{2}{*}{40.0} &
    \multirow{2}{*}{-1.771} &
    \multirow{2}{*}{0.056}\\

    $C$ &
    11 &
    2 &
    &
    &
    2 \\

  \end{tabular}
  \caption[Usage of Prototype Frequency,
           Between Groups]{%
    Frequency of using \latest{}. Comparison
    between experiment and control group for the posttest.
  }
  \label{table:up.to.date.prototype.frequency.between}
\end{table}

The data shows that the experiment respondents used the prototype more
frequently when they were logged in to \urort{} than the control respondents.
This difference is however not statistically significant.

\subsection{%
  \rp[3]{
    Does social navigation through activity streams lead users to make
    more artists on \urort{} their favorites?
  }
}

Our $H_0$ hypothesized that the amount of favorites would be greater for the
experiment group (due to their activity stream usage) in comparison to
the control group. First we looked at the differences
between the experiment and control group,
as can be seen in
\tableref{up.to.date.favorite.amount.between}.

\begin{table}
  \begin{tabular}{rrrclrrrr}

    &
    &
    &
    \multicolumn{2}{c}{$Mdn_{\sum{E}}$} \\

    &
    \multicolumn{1}{c}{$N$} &
    \multicolumn{1}{c}{$Mdn$} &
    \multicolumn{2}{c}{$- Mdn_{\sum{C}}$} &
    \multicolumn{1}{c}{$Rng$} &
    \multicolumn{1}{c}{$U$} &
    \multicolumn{1}{c}{$Z$} &
    \multicolumn{1}{c}{$p$ (1-tailed)} \\

    \cmidrule(lr){2-9}

    $E$ &
    13 &
    8 &
    \multirow{2}{*}{\twoguides} &
    \multirow{2}{*}{0} &
    49 &
    \multirow{2}{*}{68.5} &
    \multirow{2}{*}{-0.174} &
    \multirow{2}{*}{0.438}\\

    $C$ &
    11 &
    8 &
    &
    &
    38 \\

  \end{tabular}
  \caption[Number of Favorites, Between Groups]{%
    Number of favorites. Comparison
    between experiment and control group for the posttest.
  }
  \label{table:up.to.date.favorite.amount.between}
\end{table}

The data shows no notable nor significant difference between the amount of
favorites for the two groups. The range of favorites are greater for the
experiment group, but their medians are equal.

We also looked at differences in the amount of favorites within both the
control and experiment group from the pretest to the posttest. The data
is summarized in
\tableref{up.to.date.favorite.amount.within}.

\begin{table}
  \begin{whole}
  \begin{tabular}{rrrrccclrrrr}

    &
    &
    &
    &
    \multicolumn{2}{c}{Post} &
    \multicolumn{2}{c}{$Mdn_{\sum{E}}$} \\

    &
    &
    \multicolumn{1}{c}{$N$} &
    \multicolumn{1}{c}{$Mdn$} &
    \multicolumn{2}{c}{$-$ Pre} &
    \multicolumn{2}{c}{$- Mdn_{\sum{C}}$} &
    \multicolumn{1}{c}{$Rng$} &
    \multicolumn{1}{c}{$T$} &
    \multicolumn{1}{c}{$Z$} &
    \multicolumn{1}{c}{$p$ (1-tailed)} \\

    \cmidrule(lr){3-12}

    \multirow{2}{*}{$E$} &
    Pre &
    13 &
    8 &
    \multirow{2}{*}{\twoguides} &
    \multirow{2}{*}{0} &
    \multirow{4}{*}{\fourguides} &
    \multirow{4}{*}{-2} &
    50 &
    \multirow{2}{*}{39.0} &
    \multirow{2}{*}{0} &
    \multirow{2}{*}{0.515}\\

    &
    Post &
    13 &
    8 &
    &
    &
    &
    &
    49 \\

    \multirow{2}{*}{$C$} &
    Pre &
    11 &
    6 &
    \multirow{2}{*}{\twoguides} &
    \multirow{2}{*}{2} &
    &
    &
    35 &
    \multirow{2}{*}{22.0} &
    \multirow{2}{*}{-0.059} &
    \multirow{2}{*}{0.5}\\

    &
    Post &
    11 &
    8 &
    &
    &
    &
    &
    38 \\

  \end{tabular}
  \reducecaption{8}
  \caption[Number of Favorites, Within Groups]{%
    Number of favorites. Comparison
    between pretest and posttest within the experiment and control group.
  }
  \label{table:up.to.date.favorite.amount.within}
  \end{whole}
\end{table}

The within group data shows that there have been minimal development in the
amount of favorites for the experiment group after the activity stream was
used. The control group have seen an increase in the median amount of
favorites. This change is far from statistically significant.

\subsection \\

    \cmidrule(lr){1-2}

    26 &
    68.4 &
    Yes, it was an easy and quick process \\

    3 &
    7.9 &
    Yes, but I experienced small problems \\
    
    0 &
    0.0 &
    Yes, but I experienced large problems \\
    
    9 &
    23.7 &
    No\dash{}I gave up \\

  \end{tabular}
  \caption[Prototype Installation Success]{%
    Successful and non-successful installation of the prototype.
  }
  \label{table:prototype.installation.success}
\end{table}

The data shows that $26 + 3 = 29$ of $38$ respondents (76.3 \%)
to this particular survey managed to install the prototype.

Another data source for how respondents fared when trying to install our
prototype are the non-accompilish rates found in
\figurepageref{fig.experiment.mortality}. This data
shows that of the 71 ($E_1 + C_1$) respondents which bothered trying
installing the prototype, only 45 ($E_2 + C_2$) managed to do so. This equates
to 63.4\% of the participants.
\tableref{prototype.installation.drop.off} shows a comparion of the
actual non-accomplishment rates and the follow-up survey.

\begin{table}
  \begin{tabular}{rrr}

    &
    \multicolumn{1}{c}{Successfull (\%)} &
    \multicolumn{1}{c}{Failed (\%)} \\

    \cmidrule(lr){2-3}

    Follow-up questions &
    76.3 &
    23.7 \\

    Actual non-accomplishment rates &
    63.4 &
    36.6 \\

  \end{tabular}
  \caption[Prototype Installation Drop Off]{%
    Drop off for installation of the prototype.
  }
  \label{table:prototype.installation.drop.off}
\end{table}
