\section{Generalizability and Validity}

There is a few factors which have to be taken into consideration when reading
the conclusions of our empirical study which we've recently discussed.

\subsection{Scale of experiment}

As presented in 
\sectionref{empirical.results}
we saw fairly high non-accomplishment rates in our study. This meant
that we only had 25 respondents to our posttest. This number were four times
lower than what we had expected.%
\sidenote{
  See
  \sectionref{empirical.methodology.subjects}
  for details about our participation expectations.
}

\subsection{Selection of subjects}

We were only
concerned with respondents which were users of the Firefox web browser
due to the technical solution we had selected for our prototype.
There is a strong possiblity for Firefox users not beeing representative for
the \urort{} populace. Installing a custom web browser%
\sidenote{
  Unless one are using an operating system like \abbr{GNU}/Linux where
  often Firefox is the default web browser.
}
requires some technical knowledge. One could therefore argue that Firefox
users are more technical on average compared to people which use their
operating system's default browser.

\subsection{Technical seeding}

In addition to recruiting only Firefox users we may have recruited the more
technical respondents due to our complicated installation process. We have
reason to belive that the more technical respondents had a better chance to
understand and successfully complete our prototype installation process. This
means that the respondens which took our posttest have a strong possiblity of
beeing more technical apt than those respondents which only completed the
pretest.

\subsection{Motive for participation}

A fairly technical installation process and two surveys can take an
considerable amount of time to complete. We have reason to believe that only
the most active \urort{} users bothered to parttake in our experiment. Users
who often visit \urort{} are naturally more experienced in how \urort{} works
and are therefore possibly not representative for the \urort{} populace.

\subsection{Implications}

The factors we've listed are all limitiations of our study which could
interfere with the validity of our findings.
In light of these limitations we can not generalize
our results to the general \urort{} populace. The work we've
provided is early research on activity streams and prototyping with
Greasemonkey, and should be considered as preliminary indications of the
usefulness of such solutions.
