\section{Generalizability and Validity}

There is a few factors which have to be taken into consideration when reading
the conclusions of our empirical study which we've recently discussed.

\subsection{Scale of experiment}

As described in 
\sectionref{empirical.results}
we experienced quite high non-accomplishment rates in our study. This meant
that we only had 25 respondents to our posttest. This were much lower
participation numbers than we had expected%
\sidenote{
  See
  \sectionref{empirical.methodology.subjects}
  for details about our participation expectations.
}

\subsection{Selection of subjects}

% We were only concerned with Firefox users. These are probably
% not representative for Urørt users.

\subsection{Technical seeding}

% The installation process was somewhat complicated. This means
% that the most technically knowledgeable people are those
% which answered our posttests.

\subsection{Motive for participation}

% The people who want to participate in such an elaborate study is
% probably fairly interested in Urørt. They are experienced Urørt users
% and are therefore not representative of the Urørt population.

\subsection{Implications}

The factors we've listed are all limitiations of our study which could
interfere with the validity of our findings.
In light of these limitations we can not generalize
our results to the general \urort{} populace. The work we've
provided is early research on activity streams and prototyping with
Greasemonkey, and should be considered as preliminary indications of the
usefulness of such solutions.
