\section{Discussion}

This section will discuss our various research questions in relation
to the results we have presented. We'll start with looking at
several aspects of activity streams as a social navigation
mechanism before we discuss prototyping with Greasemonkey on
an established web site.

When presenting our results we worked with a level of significance
$\alpha$ of $p \leq 0.05$ (see
\sectionref{empirical.method.data.analysis.level.of.significance}
for details) and handeled values of $p$ in a black and white fashion.
\prequote[\p{1277}]{rosnow89}{argues that}{%
  surely, God loves the .06 nearly as much as the .05}
We are therefore going to discuss results with values of $p$ approaching
$\alpha$ in a more flexible fashion in this section.

Based on our profiling of the respondents (see 
\tablepageref{respondents.profile.usage} for details)
we found the two groups which were using our implementation to be
representative of the general sample in all aspects except how
frequent they used \urort{}. The control group used \urort{} more
frequent than than both the general sample and the experiment group
with a probability of 0.055 compared to the general sample.
We argue that this makes the control group more experienced
\urort{} users than the experiment group.
This is an important aspect which we'll try to keep in mind in
the discussion about activity streams which follows.

\subsection{%
  Can social navigation trough activity streams help users keep
  up-to-date on favorites' activities on \urort{}?
}

% ease of use a bad metric to compare the two versions
% as the placebo has fewer features and are therefore easier to use

\subsection{%
  Does social navigation trough activity streams lead users to more often keep
  up-to-date on favorites' activities on \urort{}?
}

\subsection{%
  Does social navigation trough activity streams lead users to make
  more artists on \urort{} their favorites?
}

\subsection{%
  Can navigational prototyping with Greasemonkey be considered a
  viable technical option when testing user behavior in an
  established web site?
}

When we conducted a study of our prototype with real world users
we got valuable feedback on how well such a system works for the average user.

\subsubsection{Limited in browser selection}

% Reference how many we had to send mail to and how many answered.
% Reference firefox usage for respondents to remove those respondents
% which did not actively use firefox from the above equation.

\subsubsection{Difficulties with installing Greasemonkey and user-scripts}

% reference non accomplish rates
% reference follow-up results (but take with a grain of salt)

\parabreak

During our development of a prototype application with Greasemonkey for
enhancing an established web page we got a feel for its pros and cons from a
development perspective.

\subsubsection{Requires no access to the established implementation}

\subsubsection{Requires little knowledge of the established implementation}

\subsubsection{Requires more work than altering the established
  implementation}

\subsubsection{Fragile when the established implementation is changed}
% Only happened once during a two month span on \urort{}. What was scary
% thought was that the changes made the user script on the client side
% obsolete. If this had happened under production usage when the user scripts
% was pushed to the clients we would be in a world of trouble. Changes on the
% server side platform can be handled more transparently.

\subsubsection{Less performant than the established implementation}

\section{Generalizability and Validity}

% We were only concerned with Firefox users. These are probably
% not representative for Firefox users.
%
% The installation process was somewhat complicated. This means
% that the most technically knowledgeable people are those
% which answered our posttests.
%
% The people who want to participate in such an elaborate study is
% probably fairly interested in Urørt. They are experienced Urørt users
% and are therefore not representative of the Urørt population.
%
% This has consequences for the validity of our study. We can not generalize
% back to Urørt users in general.
