\chapter{Implementation}
\label{chapter:implementation}

% This chapter should include design choices for my implementation. For
% example choices taken for generating relevant data for test users
% and a computational sound method for doing so. As JavaScript in browsers are
% quite inefficient it will probably be necessary to persist data at a server
% in some sort of cache. The clients could then get this data by invoking a
% single request. The result could for instance be JSON serialized. Scraping
% of such data is probably done more efficient and safer at the server side
% since multiple XMLHttpRequests in the clients for scraping and parsing could
% prove to be quite computational expensive.

As we've seen in 
\sectionref{building.on.top.of.the.web}
it's possible to build applications on top of existing web sites by creating
transparent prototype implementations. This chapter starts with an account of
what kind of navigation system we wanted to build, goes on to describe why we
decided on such navigational designs, and concludes with an explanation of how
the implementation was built\dash{}the ingredients of our implementation.

\section{Design}

\section{Architecture}

\subsection{Client Side}

\subsubsection{Programming Language}

\subsubsection{Framework}

\subsection{Server Side}

\subsubsection{Programming Language}

\subsubsection{Framework}

\subsection{Development Tools}

\subsubsection{Version Control}

\subsubsection{Editor}

A developer's main tool for authoring software is his editor. Sometimes the
implementation language warrants a specialized editor with aids for
handeling cumbersome tasks specific to that language. Such editors is often
called \emph{Integrated Development Environments}%
\sidenote{
  A good example of an IDE is
  \emph{Eclipse} (available at \url{http://eclipse.org}).
  It was first used for Java development but since extended with
  plugins for handeling other programming languages and families.
}
(IDEs) and are used most often for languages like Java and C$\sharp$.
\citet{murphy06} found that developers mostly use IDEs for navigating large
collections of source code, refactoring code, debugging code, and interacting
with revision control systems in addition to normal editor usage.

Development environments found in Lisp%
\sidenote{
  \citet[p.~69]{sandewall78} higlights the benefits of having truly
  interactive development environments:
  ``The `residential' design of programing systems, whereby all facilities
  for the user are integrated into one system with which the user communicates
  during the entire interactive session, offers great possibilities for user
  convenience.''
}
Surpassing IDEs in integration and interactivness 
A truly integrated development experience, surpassing integrated
developemnt environments, can be found in Lisp and Smalltalk's interactive
environments (cite lisp article and find smalltalk article).

The programming languages we've deciced to work with, JavaScript and Ruby, are
so expressive and dynamic in their nature that it's been argued using IDEs
with such languages can do more harm than good (citation needed). The
interactive experience provided by Lisp and Smalltalk implementations are
sadly missing from JavaScript and Ruby implementations.

We're not setteling on any editor though. Developer should use the tools
that enabeles them to most efficiently and safely interact with their code
(citation needed). The editor one chooses to use should also depend on how
much time one has available to learn to use it efficently. Truly powerfull
editors are charactericed with a big up-front investment in learning.
Therefore switching editors when one have made such investments can be quite
daunting:

\begin{citequote}{orenstein08}
  If the thought of switching editors doesn't fill you with quite a bit of
  dread, what you're using now is almost certainly underpowered, and you
  definitely haven't customized it enough.
\end{citequote}


